\documentclass{letter}

\textheight=10in
\textwidth=7.3in
\topmargin=-1.0in
\headheight=0.2in
\headsep=0.2in
\oddsidemargin=-0.5in
\evensidemargin=-0.9in
\marginparpush=1in
\marginparsep=1in
\marginparwidth=1in
\footskip=0in

\parindent=0in

\newcommand{\experience}[4]{&& \\
\textit{#1} \hspace{0.2in} & \textbf{#2} & \textbf{#3} \\
& \textit{#4} & \\}
\newcommand{\subexperience}[1]{& #1 & \\}
\newcommand{\subsubexperience}[1]{& #1 & \\}

\newcommand{\education}[3]{\hspace{0.5in}\textbf{#1, #2 (#3)}\\}
\newcommand{\subeducation}[1]{\hspace{1in}#1\\}

\begin{document}
\thispagestyle{empty}

\LARGE
\begin{center}Palmer Dabbelt\end{center}
\normalsize

\begin{tabular*}{\textwidth}{@{\extracolsep{\fill}} l r}
  Berkeley, CA& 312-725-0319 \\
  94703 & palmer@dabbelt.com \\
  \hline
\end{tabular*}\\
\vspace{0.2in}

\textbf{Experience}\\
\begin{tabular*}{\textwidth}{l l @{\extracolsep{\fill}} r}
  \experience{2016-current}{binutils Maintainer}{}{Unpaid}
  \subexperience{Maintain the RISC-V port of GNU's binutils}

  \experience{2013-2014}{CS194-24 TA}{Berkeley, CA}{TA (Class Title: Operating Systems)}
  \subexperience{The course is really "intro to Linux hacking", some QEMU development}
  \subexperience{Wrote labs, ran discussion section, wrote/graded exams}
  \subexperience{Labs: HTTP server, file system, scheduler, ethernet driver}

  \experience{2011-2013}{Tilera}{Westborough, MA}{Software Engineer}
  \subexperience{Ported the Hotspot JVM to Tilera's Tile-Gx processors}

  \experience{2009-11}{Nuvixa}{Champaign, IL}{Software Engineer}
  \subexperience{Startup doing depth assisted image processing}
  \subexperience{Optimization of code for x86 processors}
  \subexperience{Network video streaming protocol}
  
  \experience{2009-11}{ECE391 Lab TA}{Champaign, IL}{Lab TA (Class Title: Computer Systems Engineering)}
  \subexperience{Write a single user, multi tasking (with virtual memory) OS}
  \subexperience{I help students with debugging code and learn concepts}
\end{tabular*}\\
\vspace{0.2in}

\textbf{Projects}
\begin{itemize}
\item Architecture/RTL lead on a RISC-V processor on TSMC's 16nm FinFET+
  process that includes some DSP extensions.
\item RTL lead on a dual-core RISC-V processor with integrated vector units on
  ST's 28nm FDSOI process that includes integrated power management features
  and low-speed serial links.
\item Maintain and am in the process of upstreaming the RISC-V GNU toolchain
  (binutils, GCC, glibc).  I also work on the RISC-V Gentoo port, and am
  building a RISC-V DSP chip.
\item Ported the Hotspot JVM (and Sun's OpenJDK) to Tilera's Tile-Gx
  processors.  Our team consisted of two people.  I started mid-way
  through the process, with a functional port of Hotspot's interpreter
  to Tile-Pro (the previous generation of Tilera chips), but I the two
  of us ported Hotspot's interpreter and the C2 server compiler to
  Tile-Gx.
\item Created a superscalar out of order implementation of the LC3b
  processor in VHDL.  The processor has split L1 caches and a unified
  L2 cache.  It uses Tomasulu's algorithm to achieve out of order
  execution.  This was a group project which ended up winning me the
  411 design competiton.
\item Single user cluster operating system using serial interconnects
  for x86.  Written in C and x86 ASM.  In addition, for this project I
  wrote an implementation of packet-switched networking over serial
  ports, a network aware scheduler, synchronization constructs, and an
  implementation of shared memory over the network.  This project won
  me honorable mention in the ECE391 Programming Contest.
\item Created and implemented a real-time video streaming protocol
  over UDP on GNU/Linux.  Written in C.  In addition, I wrote code to
  interface with V4L USB video capture on Linux and code to interface
  with OpenGL on Linux.
\item Created and implemented a real-time background subtraction
  algorithm using depth cameras.  I wrote an implementation using C on
  GNU/Linux and am currently working on an implementation using C++ on
  Win32, which will be commercial quality code.  I have a patent on
  this algorithm.
\end{itemize}

\textbf{Awards}
\begin{itemize}
\item EE Times "Design Team of the Year" awarded to Tilera, 2012
\item $4^{\mathrm{th}}$ in ECE498 (CUDA) Programming Contest, Spring 2010
\item Won ECE411 Design Competiton, Fall 2009
\item Honorable Mention in ECE391 Programming Contest, Spring 2009
\end{itemize}
\vspace{0.2in}

\textbf{Education}\\
\begin{tabular*}{\textwidth}{l l @{\extracolsep{\fill}} r}
  \experience{August 2012 - current}{University of California}{Berkeley, CA}{PhD Computer Science}
  \experience{August 2008 - June 2011}{University of Illinois}{Champaign, IL}{BS Computer Engineering, Engineering Physics}
  \subexperience{GPA: 3.6}
\end{tabular*}
\end{document}
